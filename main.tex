\documentclass[a4paper,12pt]{article}

%========================================================================
%	Preamble Text
%========================================================================
\usepackage[english]{babel}
\usepackage[margin=1in]{geometry}
\usepackage{amsmath, amssymb}


\title{\protect\LaTeX\ Tutorial}
\author{Jeff Clark \\ Recreated by Matt Ginelli}
\date{\today}

\begin{document}
\maketitle 

\section{Introduction}
\subsection{Introduction to \LaTeX}
\LaTeX\ is a family of programs designed to produce publication-quality typeset
documents. It is particularly strong when working with mathematical symbols. \\

The history of \LaTeX\ begins with a program called \TeX. In 1978, a computer
scientist by the name of Donald Knuth grew frustrated with the mistakes that
his publishers made in typesetting his work. He decided to create a typesetting
program that everyone could easily use to typeset documents, particularly those
that include formulae, and made it freely available. The result is \TeX. \\

Knuth's product is an immensely powerful program, but one that does focus
very much on small details. A mathematician and computer scientist by the
name of Leslie Lamport wrote a variant of \TeX\ called \LaTeX\ that focuses on
document structure rather than such details. \\
 
\subsection{Required Components of a \LaTeX\ Document}
Every \LaTeX\ document must contain the following three components.  Everything else is optional (even text).

\begin{enumerate}
	\item \verb|\documentclass{article}|
	\item \verb|\begin{document}|
	\item \verb|\end{document}|
\end{enumerate}

\end{document}
